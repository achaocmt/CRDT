%!TEX root = draft.tex

\section{Implementation}
\label{sec:implementation}

In this section, we give our formation of CRDT algorithms (implementation) and their semantics.

In a CRDT algorithm, each replica stores a copy of shared data and possibly some auxiliary data. For example, in OR-set algorithm of \cite{Bieniusa:2012}, the shared data consists of a set $E$ of current item and a set $T$ of tombstone, and the auxiliary data is a counter used to ensure the uniqueness of item identifier.

Let $RepD$ be the set of data of replica. Let $Msg$ be the set of messages. Each message is of the from $msg=(dat,sorR,desR)$ and is generated since some replica does an operation. Here $dat$ is some information, %and we assume that we can use a function $op(dat)$ to obtain the operation that generates $msg$, 
$sorR \in RId$ indicates that which replica launches this operation, and $desR \in RId$ indicates the destination replica identifier of this message. %Let $data(msg)=dat$ be a function that reads the data of a message.

Then a CRDT implementation $imp$ is defined as a tuple $(Mth,Apl,InitD)$, where

\begin{itemize}
\setlength{\itemsep}{0.5pt}
\item[-] $Mth: M \times D \times PRD \times RId \rightarrow D \times PRD \times Pow(Mes)$. $Mth(m,a,d,rid) = (b,d',msgs)$ means that, if the current replica data is $d$ and we call method $m$ with argument $a$ in replica $rid$, then the resulting replica data is $d'$, we obtain return value $b$, and generate a set $msgs$ of messages. Here we assume that if $m$ is an update method, then $msgs = \{ (dat,rid,r') \vert r' \in RId \wedge rid \neq r' \}$ for some $dat$. %where $op(dat)$ uses method $m$ with argument $a$ and return value $b$.

    We also assume that, if $m(a,b)$ is a query operation, then $msgs = \emptyset$.

\item[-] $Apl: RepD \times Msg \rightarrow RepD$. $Apl(d,msg) = d'$ means that, if the current replica data is $d$ and we apply message $msg$, then we obtain replica data $d'$.

\item[-] $InitD \in RepD$ is the initial replica data.
\end{itemize}

The semantics of $imp$ is given as an LTS $\llbracket imp \rrbracket = (Q,\Sigma,\rightarrow,q_0)$ \footnote{{\color {red} For some implementations, more information may be used as tuple of state of $\llbracket imp \rrbracket$. Such information are only for make transition labels clear and do not influence execution. For example, for distributed list algorithm, the order of all known items may be used as a argument. Similarly, we may add more information into transition labels when it is necessary.}}, where

\begin{itemize}
\setlength{\itemsep}{0.5pt}
\item[-] Each state $(repD,msgs,<_{sd}) \in Q$ contains three tuples. Here $repD: RId \rightarrow RepD$ is the replica data of each replica, $msgs \subseteq Msg$ is the set of messages, $<_{sd} \subseteq Meg \times Msg$ is used to record the order of messages. $<_{sd}$ only relates messages with the same source replica identifier and with the same destination replica identifier.

Let $msgs(r,r') = \{ (\_,r,r') \in msgs \}$ denote the set of messages in $msgs$ whose source replica is $r$ and whose destination replica is $r'$. $\forall r,r' \in RId$, $<_{sd}$ is a irreflexive and acyclic total order on $msgs(r,r')$. %We also require that the transitive closure of $<_{ro}$ on operations being acyclic. 
The reason we introduce $<_{sd}$ into state is for defining simulation relation in the next section.

\item[-] $\Sigma = \{ m(a,b,r), apply(msg) \vert m \in M, a,b \in D, r \in RId, msg \in Msg \}$ is the set of transition labels.

\item[-] $\rightarrow$ is the transition relation and is defined as follows:

    \begin{itemize}
    \setlength{\itemsep}{0.5pt}
    \item[-] Doing an operation:

    $\begin{array}{l c} \bigfrac{ Mth(m,a,repD[r]) = (b,d,msgs'), msg \cap msgs' = \emptyset} {(repD,msgs,<_{sd}) {\xrightarrow{m(a,b,r)}} (repD[r:d],msgs \cup msgs',<_{sd} \odot msgs')} \end{array}$
    
    Here $<_{sd} \odot msgs'$ is obtained from $<_{sd}$ by adding  $(m_1,m_2)$, when $m_1 \in msgs$, $m_2 \in msgs'$, the souce replica of $m_1$ is the same as that of $m_2$, and the desitination replica of $m_1$ is the same as that of $m_2$. {\color {red} We also require that $msg \cap msgs' = \emptyset$. This implies that no duplicate messages will be generated.} 

    Recall that, if $m(a,b)$ is a query operation, then $msgs' = \emptyset$.

    \item[-] Applying a message:

    $\begin{array}{l c} \bigfrac{ m = (\_,\_,r) \in msgs,Apl(repD[r],m) = d } {(repD,msgs,<_{sd}) {\xrightarrow{apply(m)}} (repD[r:d],msgs - \{ m \},<_{sd}-m )} \end{array}$
    \end{itemize}

\item[-] $q_0$ is the initial state, which maps each replica identifier into $(InitD,\emptyset)$. 
\end{itemize}

We say that an execution of $\llbracket imp \rrbracket$ satisfies causal delivery, for each pair of update operations $(o_1,o_2)$, if $o_1 <_{hb} o_2$, let $(dat_1,r_1,\_)$ and $(dat_2,r_2,\_)$ be the message generated by $o_1$ and $o_2$, then for each replica $r$, the applying of $(dat_1,r_1,r)$ occurs before $(dat_2,r_2,r)$. The semantics of $imp$ with causal delivery is given as an LTS $\llbracket imp \rrbracket_{cd} = (Q_{cd},\Sigma,\rightarrow_{cd},q_{cd0})$, where

\begin {itemize}
\setlength{\itemsep}{0.5pt}
\item[-] Each state of $Q_{cd}$ is for the from $(repD,msgs,<)$, where $repD$ and $msgs$ is the same as that in $\llbracket imp \rrbracket$, and $< \subseteq (msgs \times msgs) \cup (msgs \times RId)$ is a transitive, acyclic and irreflexive relation. $<$ is used to record the happen-before relation between operations of messages. %$(m_1,m_2) \in <$ represents that the $op(m_1)$ happen before $op(m_2)$, while $(m,r) \in <$ represents that $op(m)$ is visible to replica $r$ and does not happen before any operation of replica $r$ that we could know. 
    $(m_1,m_2) \in <$ represents that the operation generating $m_1$ happens before the operation generating $m_2$, while $(m,r) \in <$ represents that the operation generating $m$ is visible to replica $r$ and does not happen before any operation of replica $r$ that we could know. 

\item[-] $\rightarrow_{cd}$ is the transition relation and contains three kinds of operations:

    The first kind is transitions of query operation. In this situation, messages and $<$ relations do not change:

    $\begin{array}{l c} \bigfrac{ Mth(m,a,repD[r]) = (b,d,\emptyset)} {(repD,msgs,<) {\xrightarrow{m(a,b,r)}} (repD[r:d],msgs,<)} \end{array}$



\item[-] The second kind is transitions of update operation :

    $\begin{array}{l c} \bigfrac{ Mth(m,a,repD[r]) = (b,d,msgs'), msgs' = \{ (dat,r,r') \vert r' \in RId \wedge r \neq r' \} \ for \ some \ dat} {(repD,msgs,<) {\xrightarrow{m(a,b,r)}} (repD[r:d],msgs \cup msgs',< \otimes dat)} \end{array}$

    Let us explain $< \otimes dat$. $< \otimes dat$ will change $<$ as follows:

    \begin{itemize}
    \setlength{\itemsep}{0.5pt}
    \item[-] $\forall k' \in RId$, in $msgs(k')$, if $m_1 < r$, then erase $(m_1,r)$, add $(m_1,(dat,r,k'))$,

    \item[-] $\forall k' \in RId$, if $m_2$ is the maximal message among $\{ (\_,r,k') \}$ w.r.t $<$, then add $(m_2,(dat,r,k'))$.

    \item[-] Make the transitive closure.
    \end{itemize}

\item[-] The third kind is transitions of applying messages:

    $\begin{array}{l c} \bigfrac{ m = (dat,r_1,r) \in msgs,Apl(repD[r],m) = d, m \ is \ minimal \ w.r.t \ < among \ messages \ in \ msgs(r) } {(repD,msgs,<) {\xrightarrow{apply(m)}} (repD[r:d],msgs - \{ m \}, < \otimes m )} \end{array}$

    Let us explain $< \otimes m$. $< \otimes m$ will change $<$ as follows:

    \begin{itemize}
    \setlength{\itemsep}{0.5pt}
    \item[-] First, remove $m$ from $<$ while keep orders of other elements in $<$ unchanged.

    \item[-] Then, $\forall k' \neq r$, if $(dat,r_1,k') \in msgs(k')$, then add $((dat,r_1,k'),k')$.

    \item[-] Make the transitive closure.
    \end{itemize}

\item[-] $q_0$ is the initial state, which maps each replica identifier into $(InitD,\emptyset,\emptyset)$.
\end{itemize}

The following lemma states that, $\llbracket imp \rrbracket_{cd}$ contains the set of causal delivery executions of $\llbracket imp \rrbracket$. Its proof can be found in Appendix \ref{sec:appendix definitions and proofs of section implementation}.

\begin{lemma}
\label{lemma:semantics of imp cd contains the set of causal delivery executions of semantics of imp}

$\llbracket imp \rrbracket_{cd} = \{ t \vert t \in \llbracket imp \rrbracket \wedge t$ satisfies causal delivery $\}$.
\end{lemma}


