%!TEX root = draft.tex

\section{Definitions and Proofs of Section \ref{sec:reference implementation}}
\label{sec:definitions and proofs of section reference implementation}



\subsection{Proof of Lemma \ref{lemma:executions of reference implementation are eventual consistent}}
\label{subsec:proof of lemma executions of reference implementation are eventual consistent}

{\noindent \bf Lemma \ref{lemma:executions of reference implementation are eventual consistent}}: $\forall t \in \llbracket RImp(Spec) \rrbracket$, $poSet(t)$ is eventual consistent w.r.t $Spec$.

\begin {proof}

Assume $RImp(Spec) = (Q,\Sigma,vis,,q_0,li,\rightarrow,livReq)$, let $t = \alpha_1, \ldots, $ and $\exists q_1,\ldots$, such that $q_0 {\xrightarrow{\alpha_1}} q_1 {\xrightarrow{\alpha_2}} \ldots$, and $livReq(t) = \textit{true}$. Let $poSet(t)=(O_t,<_t)$. Let $gi = (O_t,<_{gi})$ be the global interpretation of $t$.

For each operation $o$, assume $o$ is first introduced by a transition from $q_i$ to $q_{i+1}$, then, the local interpretation of $o$ is $li(q_i,r)$, where $r$ is the replica of $o$.

We prove this lemma by consider the four requirements individually:

\begin{itemize}
\setlength{\itemsep}{0.5pt}
\item[-] $\textit{GIpf}$: Let us consider two kinds of operations $o$.

If $\exists o'$, such that $o' <_{gi} o$, then $\exists i < j$, such that $o'$ and $o$ are the operations of $\alpha_i$ and $\alpha_j$, respectively, and one of the following cases holds: (1) $o'$ and $o$ are of same replica, or (2) $\exists i', i < i' < j$, $\alpha_{i'}=addDel(o',r')$ and $r'$ is the replica of $o$. In each case, it is easy to see that $<^{-1}_{gi}(o)$ contains lee or equal operations than the set of operations in $t[1,j]$ and its number is obviously finite.

    If $\neg \exists o'$, such that $o' <_{gi} o$, then it is easy to see that $o$ is the first operation of its replica and no operation is delivered to that replica before $o$. In this case, it is obvious that $<^{-1}_{gi}(o)$ is finite.

\item[-] $\textit{THINAIR}$: According to the definition, we could see that on each state $q_i$, local interpretation is a subset of visibility relation, and $ro$ is also a subset of visibility relation. It is easy to prove that on each state, $vis$ is acyclic.

\item[-] $\textit{RVAL}$: We only need to consider query operations for $\textit{RVAL}$ property, since only query operations have return values. According to construction of $\llbracket RImp(Spec) \rrbracket$, to launch a query operation $(m,a,b,rid,oid)$ transition of replica $r$ from state $q_i$, we already check whether $li(q_i,r)$, the local interpretation w.r.t $q_i$ and $r$, is in $Spec(m,a,b)$.

\item[-] $\textit{EVENTUAL}$: Let $P=(O_P,<_P)$ be a finite prefix of $gi$.

We can see that $<_P$ is the visibility relation of operations in $O_P$. Let $O'_P = \{ o \vert \exists o_1,o_2 \in O_P,$ $o_1$ is visible to $o_2$ via $o'_1,\ldots,o'_m$, and $o \in \{ o'_1,\ldots,o'_n \} \}$. It is easy to see that $O'_P$ is finite, and assume that operations of $O_P \cup O'_P$ are chosen before $\alpha_{idx}$

Since $livReq(t)=true$, $\exists idx1$, such that in $t[1,idx1]$, all operations in $t[1,idx]$ has been delivered to every replica. According to our construction of $li$, this implies that for each state after $\alpha_{idx1}$ and each replica (1) its local interpretation contains $O_P$ and $O'_P$, and (2) the relation of local interpretation then contains the visibility relation between operations in $O_P$.
\end{itemize}

This completes the proof of this lemma. $\qed$
\end {proof}




