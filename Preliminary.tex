%!TEX root = draft.tex

\section{Specifications and Consistencies}
\label{sec:specifications and consistencies}

In this section, we introduce our formation of specification. Then, we propose strong-return-value consistency, which is a sub-notion of eventual consistency of \cite{Bouajjani:2014} that strengthen the ``safety part'' specific for intuition of CRDT algorithms and ignores the ``liveness part''.


\subsection{Specification}
\label{subsec:specification}

Given

\begin{itemize}
\setlength{\itemsep}{0.5pt}
\item[-] $\mathbb{M}$ be a finite set of method names,

\item[-] $\mathbb{D}$ be a possibly infinite set of data domain for argument and return values,

\item[-] $\mathbb{R}$ be a finite set of replica identifiers, and

\item[-] $\mathbb{O}$ be a infinite set of operation identifiers.
\end{itemize}

%\todo{Define ``operation label'' which is $m(a,b)$ and denote operation labels by $\ell$. Operation content is not a good name.}

A operation label is a tuple $m(a)\Rightarrow b$ where $m \in \mathbb{M}$ and $a,b \in \mathbb{D}$. {\color {red}$m(a) \Rightarrow b$ represents that $m$ is called with argument $a$ and returns b. When $m$ does not use the argument (resp., return value), we write $m()\Rightarrow b$ (resp., $m(a)$) instead.} An operation $o$ is a tuple $(\ell,r,i)$, where $\ell$ is a operation label, $r \in \mathbb{R},i \in \mathbb{O}$. Let $lab(o)$ be the operation label of $o$. This definition extends naturally to the case when there are more than one arguments.

%\todo{Use $r$ instead of $rid$ and $i$ instead of $oid$. But be careful to not use $i$ in other contexts, for instance remove the $\forall i$ from the previous section. Keep $i$ and $j$ only for operation ids.}

CRDT has two kinds of method: query methods and update methods: Operations of query methods take effect only in one replica, while operations of update methods will be delivered to other replicas. A specification $Spec$ is a function that maps each operation label $\ell$ into a set of tuples $(O,<,arb)$, where $O$ is a set of update operations, $<$ is a partial order over $O$, and $arb$ is a partial order over $O \cup \{ o \}$ called arbitration order, where $lab(o)=\ell \wedge o \notin O$.
%\todo{The rest of the paragraph should be a footnote. Uninteresting details.}
%Here we require that each operation in $O \cup \{ o \}$ has unique operation identifier. Such $(O,<,<_{\mathit{arb}},l)$ tuples are called ($\Sigma$-labeled) partial-ordered set (poset, for short), where $\Sigma$ is a set of update operation contents contains that of $O$. Two labeled posets are isomorphic if there exists a bijection of operations that preserve operation contents, labels and orders. Here we require $Spec$ to be isomorphic closed: if $x \in Spec$ and $x$ and $y$ are isomorphic, then $y \in Spec$. Since the labeling function of poset is fixed, we could ignore it when the context is clear.

%\todo{I would suggest to define specifications only for query operations. I guess that you need to include $o$ in $O$ for the updates like inserting in a list. But this is kind of ugly, so I would prefer that $O$ doesn't contain $o$}

%\todo{I guess $l$ is not needed. An operation is already a label (content in your terms) with ids}

%\todo{Use $\mathit{arb}$ instead of $<_{\mathit{arb}}$. I told you several times, don't try to minimize the space occupied by your notations. And don't use complicated indices or superscripts.}

%\todo{I don't see the "deterministic" condition: for a given tuple $(O,<,arb)$, the return value is unique.}

%\todo{I think that this part about plus-minus specifications is not useful here. Give standard examples and push this discussion/examples when needed.}



Let us give several examples:

\noindent {\bf Example 1. OR-set}: Observed-Remove Set (OR-set for short) \cite{Shapiro:2011,Bieniusa:2012} contains four methods: (1) $add(a)$ inserts $a$ into set, (2) $rem(a)$ remove $a$ from set, (3) $lookup(a)\Rightarrow \mathit{true}$ (resp., $lookup(a)\Rightarrow \mathit{false}$) represents that $a$ is in set (resp., $a$ is not in set), and $elements() \Rightarrow S$ represents that the set content is $S$.

%\todo{Why do you need $\top$ as argument ?? If it's because you want to use operation labels for messages as well, I don't like it. Each part of the formalism should be as clear as possible, without unnecessary junk.}

%Let $\Sigma_{\mathit{ORS}} = \{ add(a,\top),rem(\top,a) \vert a \in \mathbb{D} \}$ be the set of contents of update operations.
The specification $S_{\mathit{ORS}}$ is as follows: (1) For $x=add(a)$ or $x = rem(a)$, $S_{\mathit{ORS}}(x)$ is the set of all tuples $(O,<,arb)$ with $arb = \emptyset$, (2) For $x=lookup(a) \Rightarrow \mathit{true}$ (resp., $x=lookup(a) \Rightarrow \mathit{false}$), $S_{\mathit{ORS}}(x)$ is the set of all tuples $(O,<,arb)$ with $arb = \emptyset$, such that the projection of operations of $add(a)$ and $rem(a)$ in $O$ contains a maximal element with operation label $add(a)$ (resp., contains no maximal element with operation label $add(a)$), and (3) $S_{\mathit{ORS}}(elements() \Rightarrow S)$ is all the tuples $(O,<,arb)$ that is in $S_{\mathit{ORS}}(lookup(a) \Rightarrow \mathit{true})$ for each $a \in S$, and is in $S_{\mathit{ORS}}(lookup(a)) \Rightarrow \mathit{false}$ for each $a \notin S$.

\noindent {\bf Example 2. Distributed list}: Distributed list has three methods: (1) $add(a,pos)$ inserts $a$ into position $pos \in \mathbb{N}$. (2) $rem(pos) \Rightarrow a$ removes the item of position $pos$, which is $a$. (3) $read() \Rightarrow l$ returns the list content.

%Let $\Sigma_{\mathit{list}} = \{ add(a,pos,\top),rem(\top,a) \vert a \in \mathbb{D},pos \in \mathbb{N} \}$ be the set of contents of update operations.
The specification $S_{\mathit{list}}$ is defined as follows: $(O,<,arb) \in S_{\mathit{list}}(\ell)$, if

\begin{itemize}
\setlength{\itemsep}{0.5pt}
\item[-] $<^{-1}$ contains finite elements, $<$ is acyclic and $arb$ is a total order of $add$ operations in $O \cup \{ o \}$, where $o \notin O$ and is in domain of $arb$.

\item[-] {\color {red}Function $f: O \cup \{ o \} \rightarrow P(O \cup \{ o \})$. For the case of $o' \in O$, let $S(o') = f(o_1) \cup \ldots \cup f(o_k)$, where $o_1,\ldots,o_k$ is the immediate predecessor of $o'$ w.r.t $<$. Then, $f(o')$ is recursively defined as 

    \begin{itemize}
    \setlength{\itemsep}{0.5pt}
    \item[-] $S(o')$, if $lab(o')=read()\Rightarrow list \wedge list = lab( arb \uparrow_{ (<^{-1}(o')-\{ x \vert (x,\_) \in S(o')\}) } )$.

    \item[-] $S(o')$, if $lab(o')=add(a,pos) \wedge ( arb \uparrow_{ (<^{-1}(o')-\{ x \vert (x,\_) \in S(o')\}) } )[pos]=o'$.

    \item[-] $S(o') \cup \{ (o_a,o') \}$, if $lab(o')=rem(pos)\Rightarrow a \wedge ( arb \uparrow_{ (<^{-1}(o')-\{ x \vert (x,\_) \in S(o')\}) } )[pos]=o_a \wedge lab(o_a)=add(a,\_)$.

    \item[-] $\mathit{Undef}$, otherwise.
    \end{itemize}

    For the case of $o$, let $S(o)$ be the union of $f(o'')$ for each $o'' \in O$, and the other part is the same as above. We require that, for each $o' \in O \cup \{ o \}$, $f(o') \neq \mathit{Undef}$.}
\end{itemize}

%\todo{Give a declarative specification of the list, like for OR-set. Descriptions which look like ``imperative programs'', e.g., "We can go through operations", "During this process".}

In our definition of distributed list specification, the arbitration order works similarly as the list order of strong list specification in \cite{Attiya:2016}.





\subsection{Consistencies}
\label{subsec:consistencies}

%\todo{Define a history as $(O,\mathit{ro})$ (again, forget about long indices), then an annotated history as $(O,\mathit{ro},\mathit{vis},\mathit{arb})$ (dont use $\mathit{mathit}$).}

A history is a tuple $(O,\mathit{ro})$, where

\begin{itemize}
\setlength{\itemsep}{0.5pt}
\item[-] $O$ is a set of operations.

\item[-] $\mathit{ro}$ is called the replica order. For each replica $r \in \mathbb{R}$, $\mathit{ro}$ is a irreflexive total order over operations with replica identifier $r$. $\mathit{ro}$ does not relate operations with different replica identifiers. We also require that for each operation $o \in O$, $\mathit{ro}^{-1}(o)$ is finite.
\end{itemize}

An annotated history is a tuple $(O,\mathit{ro},\mathit{vis},\mathit{arb})$, where

\begin{itemize}
\setlength{\itemsep}{0.5pt}
\item[-] $(O,\mathit{ro})$ is a history.

\item[-] $\mathit{vis}$ is irreflexive and acyclic, and is called the visibility order. We require that for each operation $o \in O$, $\mathit{vis}^{-1}(o)$ is finite.

\item[-] $\mathit{arb}$ is the arbitration order over update operations of $O$.
\end{itemize}

%\todo{Local interpretation meant something else in our previous paper. Use operation context for $(\mathit{vis}^{-1}(o),<,\mathit{arb}\downarrow (\mathit{vis}^{-1}(o)\times \mathit{vis}^{-1}(o)))$ where $<$ is defined as you say.}

Given an annotated history $(O,\mathit{ro},\mathit{vis},\mathit{arb})$ and an operation $o \in O$, the operation context of $o$ is a tuple $ctxt(o)=(O_o,<,arb_o)$, where

\begin{itemize}
\setlength{\itemsep}{0.5pt}
\item[-] $O_o$ is the set of update operations in $\mathit{vis}^{-1}(o)$,

\item[-] $arb_o$ is the projection of $arb$ over update operations of $O_o \cup \{ o \}$.

\item[-] $< \subseteq <_{\mathit{vis}} \uparrow_{(O_o \times O_o)}$ and it is irreflexive.

We require that, given operations $o_1,o_2,o'_1,\ldots,o'_m \in O_o$, if $(o_1,o_2) \in \mathit{vis}$ via $o'_1,\ldots,o'_m$, then $(o_1,o_2) \in <$. We say that $(o_1,o_2) \in \mathit{vis}$ via $o'_1,\ldots,o'_m$, if $(o_1,o'_1),$ $(o'_1,o'_2), \ldots, (o'_{\mathit{m-1}},o'_m),(o'_m,o_2)$ are all in $\mathit{vis}$.

\item[-] {\color {red}We require that, given $o_1,o_2 \in O$, if $\mathit{vis}^{-1}(o_1) = \mathit{vis}^{-1}(o_2)$, then $<_1 = <_2$, and $arb_1 \cup arb_2$ is acyclic, where $ctxt(o_1)=(O_1,<_1,arb_1)$ and $ctxt(o_2)=(O_2,<_2,arb_2)$.}
\end{itemize}

Let us define strong-return-value consistency (SRVC consistency, for short) as follows:

%\todo{Define ``an annotated history satisfying SRVC and then a history satisfying SRVC, i.e., there exists $\mathit{vis}$ and $\mathit{arb}$ such that the resulting annotated history satisfies SRVC.}

\begin{definition}[Strong-return-value consistency]
\label{definition:strong return value consistency}
An annotated history $(O,\mathit{ro},\mathit{vis},\mathit{arb})$ is SRVC w.r.t specification $Spec$, if there exists function $ctxt$, such that,

\begin{itemize}
\setlength{\itemsep}{0.5pt}
\item[-] $\mathit{ro} \subseteq \mathit{vis} \wedge \mathit{vis}$ is acyclic.

\item[-] $\forall o \in O$, $ctxt(o) \in Spec(lab(o))$.
\end{itemize}

A history $(O,\mathit{ro})$ is SRVC w.r.t $Spec$, if there exists $\mathit{vis}$ and $\mathit{arb}$, such that $(O,\mathit{ro},\mathit{vis},\mathit{arb})$ is SRVC consistent w.r.t $Spec$.
\end{definition}


{\color {blue}TODO:

\begin{definition}[Weak eventual consistency \cite{Bouajjani:2014}]
\label{definition:eventual consistency}
An abstract trace $t = (O,<_{\mathit{ro}},<_{\mathit{vis}},<_{\mathit{arb}})$ is called weak eventual consistency w.r.t a specification $Spec$, if there exists local interpretation $li$, such that $\mathit{THINAIR} \wedge \mathit{RVAL} \wedge \mathit{WEAKEVENTUAL}$ hold, where

\begin{itemize}
\setlength{\itemsep}{0.5pt}
\item[-] $\mathit{WEAKEVENTUAL}$: $\forall o \in O, \{ o' \vert (o,o') \notin <_{\mathit{vis}} \}$ is finite.
\end{itemize}
\end{definition}

}














\forget
{
\section{Strong Return Value Consistency}
\label{sec:strong return value consistency}

In this section, we introduce the definition of eventual consistency in \cite{Bouajjani:2014}, and related notation.


\subsection{Trace}
\label{subsec:trace}

Given a finite set $M$ of method names, a possibly infinite set $D$ of data domain, a finite set $RId=\{1,\ldots,n\}$ of replica identifiers and a possibly infinite set $OId$ of operation identifiers, an operation is a tuple $o=(m,a,b,rid,oid)$, where $m \in M$, $a,b \in D$, $rid \in RId$ and $oid \in OId$. Here $o$ represents that it calls method $m$ with argument $a$ and obtains return values $b$, and it happens on replica $rid$. This definition extends naturally to the case when there are more than one arguments.

A partial-ordered set (poset, for short) is a tuple $(A,<)$ contains a possibly infinite set $A$ and a partial-order $<$ of $A$. A trace $\tau$ is a possibly infinite poset $(Os,ro)$, where
\begin{itemize}
\setlength{\itemsep}{0.5pt}
\item[-] $Os$ is a set of operations. Here we assume that a trace does not contain two operations with the same identifier.

\item[-] $ro$ is called the replica order. For each replica identifier $k$, $ro$ is a irreflexive total order over operations with replica identifier $k$. It does not relate operations with different replica identifiers. We also require that for each $o \in Os$, $ro^{-1}(o)$ is a finite set.
\end{itemize}

A poset $(A_1,<_1)$ is called a prefix of a poset $(A_2,<_2)$, denoted by $(A_1,<_1) \preceq (A_2,<_2)$, if $A_1 \subseteq A_2$, $<_1$ is the intersection of $<_2$ and $A_1 \times A_1$, and $A_1 = <_2^{-1}(A)$. A $\Sigma$-labeled poset is a tuple $(A,<,l)$, where $(A,<)$ is a poset and $l:A \rightarrow \Sigma$ is a function that labels each element of $A$ with an element in $\Sigma$. The set of all $\Sigma$-labeled poset is denoted as $PoSet_{\Sigma}$.


\subsection{Eventual Consistency}
\label{subsec:eventual consistency}

For a set $A$, let $P(A)$ denotes the set of all subsets of $A$. Let $\Sigma(M,D) = \{ m(a,b) \vert m \in M, a,b, \in D \}$. A specification $Spec$ is a function $Spec: \Sigma(M,D) \rightarrow P(PoSet_{\Sigma(M,D)})$.

Given a trace $\tau = (Os,ro)$, for each operation $o \in Os$, its local interpretation, denoted $li(o) = (O_o,<_o)$, is a poset of operations. The local interpretation defines another relation between operations, called executed-before and denoted by $eb$. $(o,o') \in eb$, if $o'$ is in the local interpretation of $o$. We say that the return value of an operation $o = (m,a,b,rid,oid) \in Os$ is correct, if the labeled poset defined by $li(o)$, where every operation $(m',a',b',rid',oid')$ is labeled by $(m',a',b')$, belongs to $Spec(m(a,b))$. Then, a trace $\tau$ is safe if the return value of every operations of $\tau$ is correct.

Given trace $\tau$, its global interpretation is a partial-order over all the operations of $\tau$.

\begin{definition}[Eventual Consistency \cite{Bouajjani:2014}]
\label{definition:eventual consistency}
A trace $\tau = (Os,ro)$ is called eventual consistency w.r.t a specification $Spec$, if:

$\exists gi$ an irreflexive partial order over $Os$, such that $\forall o \in Os$, $\exists li[o]$ an irreflexive poset, such that

$\mathit{GIpf} \wedge \mathit{THINAIR} \wedge \mathit{RVAL} \wedge \mathit{EVENTUAL}$ hold.
\end{definition}

Here, $GIpf$, $THINAIR$, $RVAL$ and $EVENTUAL$ is given as follows:

\begin{itemize}
\setlength{\itemsep}{0.5pt}
\item[-] $\mathit{GIpf}$: $\forall o \in Os$, $<^{-1}_{gi}(o)$ is finite.

\item[-] $\mathit{THINAIR}$: $eb \cup ro$ is acyclic.

\item[-] $\mathit{RVAL}$: $\forall o = (m,a,b,rid,oid) \in Os$, $li_{\Sigma(M,D)}(o) \in Spec(m,a,b)$. Here $li_{\Sigma(M,D)}(o)$ is obtained from $li(o)$ by labeling each $(m',a',b',rid',oid')$ with $(m,a,b)$.

\item[-] $\mathit{EVENTUAL}$: For any finite prefix $P$ of the poset $(Os,gi)$, $\{ o \vert o \in Os, P \npreceq li(o) \}$ is finite.
\end{itemize}
}
