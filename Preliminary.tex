%!TEX root = draft.tex

\section{Convergent Return-Value Consistency}
\label{sec:specifications and consistencies}

%In this section, we introduce our formation of specification. Then, we propose strong-return-value consistency, which is a sub-notion of eventual consistency of \cite{Bouajjani:2014} that strengthen the ``safety part'' specific for intuition of CRDT algorithms and ignores the ``liveness part''.




%\subsection{Specification}
%\label{subsec:specification}

%Given

%\begin{itemize}
%\setlength{\itemsep}{0.5pt}
%\item[-] $\mathbb{M}$ be a finite set of method names,

%\item[-] $\mathbb{D}$ be a possibly infinite set of data domain for argument and return values,

%\item[-] $\mathbb{R}$ be a finite set of replica identifiers, and

%\item[-] $\mathbb{O}$ be a infinite set of operation identifiers.
%\end{itemize}

%\todo{Define ``operation label'' which is $m(a,b)$ and denote operation labels by $\ell$. Operation content is not a good name.}

Given a finite set $\mathbb{M}$ of method names, a possibly infinite set $\mathbb{D}$ of data domain for argument and return values, a finite set $\mathbb{R}$ of replica identifiers, and a infinite set $\mathbb{O}$ of operation identifiers. A operation label is a tuple $m(a)\Rightarrow b$ where $m \in \mathbb{M}$ and $a,b \in \mathbb{D}$. $m(a) \Rightarrow b$ represents that $m$ is called with argument $a$ and returns b. When $m$ does not use the argument (resp., return value), we write $m()\Rightarrow b$ (resp., $m(a)$) instead. An operation $o$ is a tuple $(\ell,r,i)$, where $\ell$ is a operation label, $r \in \mathbb{R},i \in \mathbb{O}$. Let $lab(o)$ be the operation label of $o$. This definition extends naturally to the case when there are more than one arguments.

A history is a tuple $(O,\mathit{ro})$, where $O$ is a set of operations, and $\mathit{ro}$ is called the replica order. For each replica $r \in \mathbb{R}$, $\mathit{ro}$ is a irreflexive total order over operations with replica identifier $r$. $\mathit{ro}$ does not relate operations with different replica identifiers. %We also require that for each operation $o \in O$, $\mathit{ro}^{-1}(o)$ is finite.

CRDT has two kinds of method: query methods and update methods. An annotated history is a tuple $(O,\mathit{vis},\mathit{arb})$, where

\begin{itemize}
\setlength{\itemsep}{0.5pt}
\item[-] $O$ is a set of operations.

\item[-] $\mathit{vis}$ is a irreflexive and acyclic relation over $O$, and is called the visibility relation. We require that for each operation $o \in O$, $\mathit{vis}^{-1}(o)$ is finite.

\item[-] $\mathit{arb}$ is a partial order over update operations of $O$ and is called the arbitration order.
\end{itemize}

A operation context of a operation $o$ is a tuple $(O,<,\mathit{arb})$, where $O$ is a set of update operations, $<$ is a relation over $O$, $o \notin O$, and $\mathit{arb}$ is a partial order over update operations of $O \cup \{ o \}$. A specification $Spec$ is a function that maps each operation label $\ell$ into a set of elements, while each of them is a operation context $(O,<,\mathit{arb})$ for some operation $o$ with $lab(o) = \ell$. A specification is deterministic, if there does not exists query method $m$ and tuple $(O,<,\mathit{arb})$, such that $(O,<,\mathit{arb}) \in Spec(m(a) \Rightarrow b)$, $(O,<,\mathit{arb}) \in Spec(m(a') \Rightarrow b')$, and $a \neq a' \vee b \neq b'$. From now on, we consider only deterministic specifications.

Given an annotated history $(O,\mathit{vis},\mathit{arb})$ and an operation $o \in O$, the operation context of $o$ is a obtained by a tuple $ctxt(o)=(O',<,\mathit{arb}')$, where $O'$ is the set of update operations in $\mathit{vis}^{-1}(o)$, $<$ is the projection of $\mathit{vis}$ over $O'$, and $\mathit{arb}'$ is the projection of $\mathit{arb}$ over update operations of $O' \cup \{ o \}$. Therefore, given an annotated history $(O,\mathit{vis},\mathit{arb})$ and $o_1,o_2 \in O$, if $\mathit{vis}^{-1}(o_1) = \mathit{vis}^{-1}(o_2)$, then $O_1 = O_2$ and $<_1 = <_2$, where $ctxt(o_1)=(O_1,<_1,\mathit{arb}_1)$ and $ctxt(o_2)=(O_2,<_2,\mathit{arb}_2)$. This comply with the commutativity intuition of CRDT algorithm.

%To comply with the commutativity intuition of CRDT algorithm, we further require that, given an annotated history $(O,\mathit{vis},\mathit{arb})$ and $o_1,o_2 \in O$, if $\mathit{vis}^{-1}(o_1) = \mathit{vis}^{-1}(o_2)$, then $<_1 = <_2$, where $ctxt(o_1)=(O_1,<_1,\mathit{arb}_1)$ and $ctxt(o_2)=(O_2,<_2,\mathit{arb}_2)$.

The notion of Convergent Return-Value Consistency (CRVC, for short) is defined as follows:
%\todo{Define ``an annotated history satisfying SRVC and then a history satisfying SRVC, i.e., there exists $\mathit{vis}$ and $\mathit{arb}$ such that the resulting annotated history satisfies SRVC.}

\begin{definition}[Convergent Return-Value Consistency]
\label{definition:strong return value consistency}
An annotated history $(O,\mathit{vis},\mathit{arb})$ is CRVC w.r.t specification $Spec$, if there exists function $ctxt$, such that,

\begin{itemize}
\setlength{\itemsep}{0.5pt}
\item[-] $\mathit{vis}$ is acyclic.

\item[-] $\forall o \in O$, $ctxt(o) \in Spec(lab(o))$.
\end{itemize}

A history $(O,\mathit{ro})$ is CRVC w.r.t $Spec$, if there exists $\mathit{vis}$ and $\mathit{arb}$, such that $\mathit{ro} \subseteq \mathit{vis}$, and $(O,\mathit{ro},\mathit{vis},\mathit{arb})$ is CRVC consistent w.r.t $Spec$.
\end{definition}

\noindent {\bf Example 1. OR-set}: Observed-Remove Set (OR-set for short) \cite{Shapiro:2011,Bieniusa:2012} contains four methods: (1) $add(a)$ inserts $a$ into set, (2) $rem(a)$ remove $a$ from set, (3) $lookup(a)\Rightarrow \mathit{true}$ (resp., $lookup(a)\Rightarrow \mathit{false}$) represents that $a$ is in set (resp., $a$ is not in set), and $elements() \Rightarrow S$ represents that the set content is $S$.

%\todo{Why do you need $\top$ as argument ?? If it's because you want to use operation labels for messages as well, I don't like it. Each part of the formalism should be as clear as possible, without unnecessary junk.}

%Let $\Sigma_{\mathit{ORS}} = \{ add(a,\top),rem(\top,a) \vert a \in \mathbb{D} \}$ be the set of contents of update operations.
The specification $S_{\mathit{ORS}}$ is as follows: (1) For $x=add(a)$, $S_{\mathit{ORS}}(x)$ is the set of all tuples $(O,<,\emptyset)$, (2) For $x=lookup(a) \Rightarrow \mathit{true}$ (resp., $x=lookup(a) \Rightarrow \mathit{false}$), $S_{\mathit{ORS}}(x)$ is the set of all tuples $(O,<,\emptyset)$, such that the projection of $<$ over $add(a)$ and $rem(a)$ in $O$ contains a maximal element with operation label $add(a)$ (resp., contains no maximal element with operation label $add(a)$), (3) For $x = rem(a)$, $S_{\mathit{ORS}}(x)$ is the same as $x = lookup(a) \Rightarrow \mathit{true}$, and (4) $S_{\mathit{ORS}}(elements() \Rightarrow S)$ is all the tuples $(O,<,\emptyset)$ that is in $S_{\mathit{ORS}}(lookup(a) \Rightarrow \mathit{true})$ for each $a \in S$, and is in $S_{\mathit{ORS}}(lookup(a)) \Rightarrow \mathit{false}$ for each $a \notin S$. 

\noindent {\bf Example 2. Distributed list}: Distributed list has three methods: (1) $add(a,pos)$ inserts identifier $a$ into position $pos \in \mathbb{N}$, (2) $rem(a)$ removes the identifier $a$, and (3) $read() \Rightarrow l$ returns the list content. Here intuitively, we assume that each identifier are putted at most once globally.

%Let $\Sigma_{\mathit{list}} = \{ add(a,pos,\top),rem(\top,a) \vert a \in \mathbb{D},pos \in \mathbb{N} \}$ be the set of contents of update operations.
The specification $S_{\mathit{list}}$ is defined as follows: $(O,<,\mathit{arb}) \in S_{\mathit{list}}(\ell)$, if

\begin{itemize}
\setlength{\itemsep}{0.5pt}
\item[-] $<^{-1}$ contains finite elements, $<$ is acyclic and $arb$ is a total order of $add$ operations in $O \cup \{ o \}$, where $o \notin O$ and is in domain of $arb$.

\item[-] Let $R = \{ i \vert \exists b, (add(b),\_,i),(rem(b),\_,\_) \in O \}$ and $s = \mathit{arb} \uparrow_{ O \cup \{ o \} \setminus R }$. Then,

    \begin{itemize}
    \setlength{\itemsep}{0.5pt}
    \item[-] Either $\ell = add(a,pos) \wedge s[pos] = o$,

    \item[-] Or $\ell = rem(a) \wedge (add(a),\_,\_) \in O$,

    \item[-] Or $\ell = read() \Rightarrow l \wedge l = lab(s)$.
    \end{itemize}
\end{itemize}

%\todo{Give a declarative specification of the list, like for OR-set. Descriptions which look like ``imperative programs'', e.g., "We can go through operations", "During this process".}

Here we essentially use the ist order of strong list specification in \cite{Attiya:2016} as the arbitration order.

%Given an annotated history $(O,\mathit{vis},\mathit{arb})$ and an operation $o$, $ctxt(o)=(O',<,\mathit{arb}')$ of list is defined as follows: $< = \mathit{vis} \uparrow_{(O' \times O')}$. 
{\color {red}Remark: For OR-set, the operation context of annotated history is more complicated: Given an annotated history $(O,\mathit{vis},\mathit{arb})$ and an operation $o$, $ctxt(o)=(O',<,\mathit{arb}')$, where $O'$ and $\mathit{vis}'$ is as before, and $< = \mathit{vis} \uparrow_{(O' \times O')} - \{ (o_1,o_2) \vert o_2 \notin FstRem(O,\mathit{vis},o_1), \exists o_3 \in O, (o_1,o_3), (o_3,o_2),(o_1,o_2) \in \mathit{vis}, o_1,o_2 \in O', o_3 \notin O' \}$. Here $FstRem(O,\mathit{vis},o)$ is the set of first visible matching remove of $o$ w.r.t $\mathit{vis}$, and is defined as $FstRem(O,\mathit{vis},o) = \{o' \vert lab(o')=rem(a), (o,o') \in \mathit{vis}, \neg \exists o'',  lab(o'') = rem(a)  \wedge (o,o''),(o'',o') \in \mathit{vis} \}$ for $lab(o) = add(a)$.} 


%Given an annotated history $(O,\mathit{vis},\mathit{arb})$ and an operation $o$, $ctxt(o)=(O',<,\mathit{arb}')$ of OR-set is defined as follows: $\mathit{arb}' = \emptyset$, and $< = \mathit{vis} \uparrow_{(O' \times O')} - \{ (o_1,o_2) \vert o_2 \notin Minus(O,\mathit{vis},o_1), \exists o_3 \in O, (o_1,o_3), (o_3,o_2),(o_1,o_2) \in \mathit{vis}, o_1,o_2 \in O', o_3 \notin O' \}$. Here $Minus(O,\mathit{vis},o)$ is the set of first visible matching remove of $o$ w.r.t $\mathit{vis}$, and is defined as $Minus(O,\mathit{vis},o) = \{o' \vert lab(o')=rem(a), (o,o') \in \mathit{vis}, \neg \exists o'',  lab(o'') = rem(a)  \wedge (o,o''),(o'',o') \in \mathit{vis} \}$ for $lab(o) = add(a)$.




%The specification $S_{\mathit{list}}$ is defined as follows: $(O,<,arb) \in S_{\mathit{list}}(\ell)$, if

%\begin{itemize}
%\setlength{\itemsep}{0.5pt}
%\item[-] $<^{-1}$ contains finite elements, $<$ is acyclic and $arb$ is a total order of $add$ operations in $O \cup \{ o \}$, where $o \notin O$ and is in domain of $arb$.

%\item[-] {\color {red}Function $f: O \cup \{ o \} \rightarrow P(O \cup \{ o \})$. For the case of $o' \in O$, let $S(o') = f(o_1) \cup \ldots \cup f(o_k)$, where $o_1,\ldots,o_k$ is the immediate predecessor of $o'$ w.r.t $<$. Then, $f(o')$ is recursively defined as

%    \begin{itemize}
%    \setlength{\itemsep}{0.5pt}
%    \item[-] $S(o')$, if $lab(o')=read()\Rightarrow list \wedge list = lab( arb \uparrow_{ (<^{-1}(o')-\{ x \vert (x,\_) \in S(o')\}) } )$.

%    \item[-] $S(o')$, if $lab(o')=add(a,pos) \wedge ( arb \uparrow_{ (<^{-1}(o')-\{ x \vert (x,\_) \in S(o')\}) } )[pos]=o'$.

%    \item[-] $S(o') \cup \{ (o_a,o') \}$, if $lab(o')=rem(pos)\Rightarrow a \wedge ( arb \uparrow_{ (<^{-1}(o')-\{ x \vert (x,\_) \in S(o')\}) } )[pos]=o_a \wedge lab(o_a)=add(a,\_)$.

%    \item[-] $\mathit{Undef}$, otherwise.
%    \end{itemize}

%    For the case of $o$, let $S(o)$ be the union of $f(o'')$ for each $o'' \in O$, and the other part is the same as above. We require that, for each $o' \in O \cup \{ o \}$, $f(o') \neq \mathit{Undef}$.}
%\end{itemize}

%\todo{Give a declarative specification of the list, like for OR-set. Descriptions which look like ``imperative programs'', e.g., "We can go through operations", "During this process".}

%In our definition of distributed list specification, the arbitration order works similarly as the list order of strong list specification in \cite{Attiya:2016}.














%\todo{Use $r$ instead of $rid$ and $i$ instead of $oid$. But be careful to not use $i$ in other contexts, for instance remove the $\forall i$ from the previous section. Keep $i$ and $j$ only for operation ids.}

%CRDT has two kinds of method: query methods and update methods: Operations of query methods take effect only in one replica, while operations of update methods will be delivered to other replicas.

%A specification $Spec$ is a function that maps each operation label $\ell$ into a set of tuples $(O,<,arb)$, where $O$ is a set of update operations, $<$ is a partial order over $O$, and $arb$ is a partial order over $O \cup \{ o \}$ called arbitration order, where $lab(o)=\ell \wedge o \notin O$.

%We require that, given $o_1,o_2 \in O$, if $\mathit{vis}^{-1}(o_1) = \mathit{vis}^{-1}(o_2)$, then $<_1 = <_2$, and $arb_1 \cup arb_2$ is acyclic, where $ctxt(o_1)=(O_1,<_1,arb_1)$ and $ctxt(o_2)=(O_2,<_2,arb_2)$.



%\todo{The rest of the paragraph should be a footnote. Uninteresting details.}
%Here we require that each operation in $O \cup \{ o \}$ has unique operation identifier. Such $(O,<,<_{\mathit{arb}},l)$ tuples are called ($\Sigma$-labeled) partial-ordered set (poset, for short), where $\Sigma$ is a set of update operation contents contains that of $O$. Two labeled posets are isomorphic if there exists a bijection of operations that preserve operation contents, labels and orders. Here we require $Spec$ to be isomorphic closed: if $x \in Spec$ and $x$ and $y$ are isomorphic, then $y \in Spec$. Since the labeling function of poset is fixed, we could ignore it when the context is clear.

%\todo{I would suggest to define specifications only for query operations. I guess that you need to include $o$ in $O$ for the updates like inserting in a list. But this is kind of ugly, so I would prefer that $O$ doesn't contain $o$}

%\todo{I guess $l$ is not needed. An operation is already a label (content in your terms) with ids}

%\todo{Use $\mathit{arb}$ instead of $<_{\mathit{arb}}$. I told you several times, don't try to minimize the space occupied by your notations. And don't use complicated indices or superscripts.}

%\todo{I don't see the "deterministic" condition: for a given tuple $(O,<,arb)$, the return value is unique.}

%\todo{I think that this part about plus-minus specifications is not useful here. Give standard examples and push this discussion/examples when needed.}





%\subsection{Consistencies}
%\label{subsec:consistencies}

%\todo{Define a history as $(O,\mathit{ro})$ (again, forget about long indices), then an annotated history as $(O,\mathit{ro},\mathit{vis},\mathit{arb})$ (dont use $\mathit{mathit}$).}

%A history is a tuple $(O,\mathit{ro})$, where

%\begin{itemize}
%\setlength{\itemsep}{0.5pt}
%\item[-] $O$ is a set of operations.

%\item[-] $\mathit{ro}$ is called the replica order. For each replica $r \in \mathbb{R}$, $\mathit{ro}$ is a irreflexive total order over operations with replica identifier $r$. $\mathit{ro}$ does not relate operations with different replica identifiers. We also require that for each operation $o \in O$, $\mathit{ro}^{-1}(o)$ is finite.
%\end{itemize}

%An annotated history is a tuple $(O,\mathit{ro},\mathit{vis},\mathit{arb})$, where

%\begin{itemize}
%\setlength{\itemsep}{0.5pt}
%\item[-] $(O,\mathit{ro})$ is a history.

%\item[-] $\mathit{vis}$ is irreflexive and acyclic, and is called the visibility order. We require that for each operation $o \in O$, $\mathit{vis}^{-1}(o)$ is finite.

%\item[-] $\mathit{arb}$ is the arbitration order over update operations of $O$.
%\end{itemize}

%\todo{Local interpretation meant something else in our previous paper. Use operation context for $(\mathit{vis}^{-1}(o),<,\mathit{arb}\downarrow (\mathit{vis}^{-1}(o)\times \mathit{vis}^{-1}(o)))$ where $<$ is defined as you say.}

%Given an annotated history $(O,\mathit{ro},\mathit{vis},\mathit{arb})$ and an operation $o \in O$, the operation context of $o$ is a tuple $ctxt(o)=(O_o,<,arb_o)$, where

%\begin{itemize}
%\setlength{\itemsep}{0.5pt}
%\item[-] $O_o$ is the set of update operations in $\mathit{vis}^{-1}(o)$,

%\item[-] $arb_o$ is the projection of $arb$ over update operations of $O_o \cup \{ o \}$.

%\item[-] $< \subseteq <_{\mathit{vis}} \uparrow_{(O_o \times O_o)}$ and it is irreflexive.

%We require that, given operations $o_1,o_2,o'_1,\ldots,o'_m \in O_o$, if $(o_1,o_2) \in \mathit{vis}$ via $o'_1,\ldots,o'_m$, then $(o_1,o_2) \in <$. We say that $(o_1,o_2) \in \mathit{vis}$ via $o'_1,\ldots,o'_m$, if $(o_1,o'_1),$ $(o'_1,o'_2), \ldots, (o'_{\mathit{m-1}},o'_m),(o'_m,o_2)$ are all in $\mathit{vis}$.

%\item[-] {\color {red}We require that, given $o_1,o_2 \in O$, if $\mathit{vis}^{-1}(o_1) = \mathit{vis}^{-1}(o_2)$, then $<_1 = <_2$, and $arb_1 \cup arb_2$ is acyclic, where $ctxt(o_1)=(O_1,<_1,arb_1)$ and $ctxt(o_2)=(O_2,<_2,arb_2)$.}
%\end{itemize}

%Let us define strong-return-value consistency (SRVC consistency, for short) as follows:

%\todo{Define ``an annotated history satisfying SRVC and then a history satisfying SRVC, i.e., there exists $\mathit{vis}$ and $\mathit{arb}$ such that the resulting annotated history satisfies SRVC.}

%\begin{definition}[Strong-return-value consistency]
%\label{definition:strong return value consistency}
%An annotated history $(O,\mathit{ro},\mathit{vis},\mathit{arb})$ is SRVC w.r.t specification $Spec$, if there exists function $ctxt$, such that,

%\begin{itemize}
%\setlength{\itemsep}{0.5pt}
%\item[-] $\mathit{ro} \subseteq \mathit{vis} \wedge \mathit{vis}$ is acyclic.

%\item[-] $\forall o \in O$, $ctxt(o) \in Spec(lab(o))$.
%\end{itemize}

%A history $(O,\mathit{ro})$ is SRVC w.r.t $Spec$, if there exists $\mathit{vis}$ and $\mathit{arb}$, such that $(O,\mathit{ro},\mathit{vis},\mathit{arb})$ is SRVC consistent w.r.t $Spec$.
%\end{definition}














