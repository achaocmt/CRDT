%!TEX root = draft.tex
\section{Introduction}
\label{sec:introduction}

Conflict-Free Replicated Data Types (CRDTs)~\cite{Shapiro:2011} have
emerged as a most promising solution to the problem of ensuring
some degree of consistency in highly-available distributed systems.
%
The key idea is that while a CRDT might not be consistent at all times
through the whole system during an execution, the state of the
data type will \emph{eventually} converge to a unique state (c.f.
eventual consistency~\cite{Burckhardt:2014b}).
%
All the while, all participants of the systems can issue operations
which are immediately performed in a single replica without need to
synchronize with other replicas, thus achieving high-availability,
even under partitions.

While CRDTs offer great advantages in terms of performance and
availability, their semantics and implementation are far from simple,
often involving sophisticated specifications~\cite{Burckhardt:2014b}, and
even more involved implementations~\cite{Shapiro:2011}.
%
As a testament to this claim, we remark that the discovery of a single
CRDT implementation often deserves a paper on itself~\cite{RohJKL11,KleppmannB17}.
%
A consequence of the complexity of CRDT implementations is that it is
often hard to gain confidence on their correctness.
%
In this paper we consider a generic methodology to the verification of
CRDTs implementing collection data structures.

% \fxnote*{Not sure} {
%   Some of the most important data types in
%   distributed programming are lists, tables, etc.
% }
We consider the problem of formally verifying that a certain
implementation of a CRDT satisfies a specification given in the
style of~\cite{Burckhardt:2014}.
%
Unlike prior approaches to this problem we proceed by
proving simulations between the implementation, and operational
encodings of the specification.
%
Critically, for our simulation proofs to carry over, the
specifications need to include enough information about when each
operation is made visible to other replicas participating in the
system.
%
This is generally called the message delivery policy~\cite{}.
%
We show how to encode the required delivery policy in the
specification in order to facilitate the proof establishing that the
delivery policy of the implementation respects our expectation on
delivery. 


% \fxnote*{GP:}{It would be nice to have a simple example of spec/imp
%   here to illustrate the difficulty.}

% \fxwarning*{Complete}{what is it that is remarkably new here?}.
% \fxwarning{Talk about + - operations?}

In a nutshell our methodology is based on the following concepts and
contributions:
\begin{itemize}[$\bullet$]
\item Starting from a specification in the style of~\cite{Burckhardt:2014}
  we provide a reference operational semantics of the CRDT which is
  then used as the definitive correctness condition for any
  implementation of the data structure;
\item our operational reference implementation incorporates in the
  state enough information about the history of the state in the
  current execution;
\item we incorporate in the specification events to extend the
  visibility of operations to different replicas. These events roughly
  correspond to the delivery of messages in the implementation and
  they enable to consider different message delivery policies;
\item we then show how we can simplify the specification reference
  semantics by removing unnecessary information from states whenever
  possible. 
  % 
  This step can greatly reduce the complexity of the specifications
  and thus be instrumental in the proofs to be carried out;
\item we take well known implementations of sophisticated CRDTs:
  RGA~\cite{RohJKL11}, and two OR-Set
  implementations~\cite{Shapiro:2011, ,Bieniusa:2012} and show how
  simulation relations (w.r.t. the reference semantics
  above-mentioned) can be easily built for each of them;
\item we show that these simulations actually imply the correctness of
  the implementations.
  %
  This conclusively shows the conformance of the implementations with
  the specification.
\end{itemize}

While this paper deals only with a few CRDT implementations, we
believe that the approach is general enough to cover most existing
CRDTs.
%

%%% Local Variables:
%%% mode: latex
%%% TeX-master: "draft"
%%% End:
