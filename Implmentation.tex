%!TEX root = draft.tex

\section{CRDT Implementations}
\label{sec:implementation}

CRDT implementations can be viewed as communicating state machines where each state machine models the behavior of one replica. Each operation submitted to a replica updates the local state of that replica before being broadcasted to all the other replicas, in the case of update operations.

Let $\mathbb{L}$ be the set of replica states and $\mathbb{MSG}$ the set of messages. Each message is of the form $(d,r_s,r_d)$ where $d$ is the payload, $r_s \in \mathbb{R}$ is the identifier of the sender, and $r_d \in \mathbb{R}$ is the identifier of the receiver. A CRDT implementation $\mathit{Imp}$ is defined as a tuple $(\rightarrow_o,\rightarrow_m,d_0,f_{\mathit{arb}}, \Sigma_e)$, where

\begin{itemize}
\setlength{\itemsep}{0.5pt}
\item[-] $\rightarrow_o: \mathbb{L} \times \mathbb{M} \times \mathbb{D} \times \mathbb{R} \rightarrow \mathbb{L} \times \mathbb{D} \times 2^{\mathbb{MSG}})$ shows how operations update the local state of the replica and generate messages (to be broadcasted to the other replicas). A transition $(d,m,a,r) \rightarrow_o (d',b,\mathit{mgs})$ means that an invocation of $m$ with input $a$ updates the replica state from $d$ to $d'$, returns value $b$, and generates the set of messages $\mathit{mgs}$.
%represents that, when the local state of replica $r$ is $d$, if we call method $m$ with argument $a$ in replica $r$, then we obtain return value $b$, change the local state of replica $r$ into $d'$, and generate a set $\mathit{mgs}$ of messages. Here $\mathit{Pow}(\mathbb{MSG})$ is the power set of $\mathbb{MSG}$.
If $m \in \mathbb{U}$, then $\mathit{mgs} = \{ (x,r,r') \vert r' \in \mathbb{R} \wedge r \neq r' \}$ for some payload $x$. Otherwise, $\mathit{mgs} = \emptyset$.
%The violation of pre-condition of method is modeled by mapping to $\mathit{undef}$.

\item[-] $\rightarrow_m: \mathbb{L} \times \mathbb{R} \times \mathbb{MSG} \rightarrow \mathbb{L}$ shows how the replica state is updated when a message is received. A transition $(d,r,\mathit{msg}) \rightarrow_m d'$ means that ``applying'' the message $\mathit{msg}$ updates the replica state from $d$ to $d'$
% if the local state of replica $r$ is $d$ and we apply message $\mathit{msg}$, then we obtain local state $d'$. The violation of pre-condition of applying a message is modeled by mapping to $\mathit{undef}$.

\item[-] $d_0 \in \mathbb{L}$ is the initial local state.

\item[-] the function $f_{\mathit{arb}}$ takes the local states of all replicas and a set of messages, and returns a total order over a set of operations in $\Sigma_e$. %Let $\Sigma_e$ be the set of such elements.
%
    %To do this, we need a method to extract elements from local states and messages. Let the set of extracted elements to be $\Sigma_e$.
%
%\item[-] $f_{o}: \Sigma_{\mathtt{RD}}^{\vert \mathbb{R} \vert} \times Pow(\mathbb{MSG}) \Rightarrow Pow(\Sigma_o)$, where $\Sigma_o$ is a set of elements and each element represents an operation. Given replica data of each replica, $f_{o}$ is used to obtain the ``operations'' known by current state of implementation.
%
%\item[-] $f_{arb}: \Sigma_{\mathtt{RD}}^{\vert \mathbb{R} \vert} \times Pow(\mathbb{MSG}) \Rightarrow \Sigma_o^*$. Given replica data of each replica and messages, $f_{arb}$ is used to obtain the arbitration relation.
\end{itemize}


The semantics of $\mathit{Imp}$ is given as an LTS denoted $[|\mathit{Imp}|] = (Q,\Sigma,\rightarrow,q_0)$, where

\begin{itemize}
\setlength{\itemsep}{0.5pt}
\item[-] Each state $(\mathit{data},\mathit{msgs}) \in Q$ contains a tuple of replica states $\mathit{data}: \mathbb{R} \rightarrow \mathbb{L}$ and a set of messages $\mathit{msgs} \subseteq \mathbb{MSG}$ in transit (not yet received).

\item[-] $\Sigma$ contains two kinds of transition labels: $m(a,b,r,\mathit{arb})$ represents that a $m(a) \Rightarrow b$ operation is done in replica $r$ and the arbitration order is changed into $\mathit{arb}$, and $\mathit{apply}(\mathit{msg})$ means applying message $\mathit{msg}$.

\item[-] the transition relation $\rightarrow$ is defined as follows:
  \begin{center}
% \medskip
$\begin{array}{l c}
   \bigfrac{ 
   \begin{array}{c}
     (\mathit{data}(r),m,a,r) \rightarrow_o (d,b,\mathit{mgs}')\\ 
     \mathit{msgs} \cap \mathit{mgs}' = \emptyset\quad f_{\mathit{arb}}((\mathit{data}[r:= d], \mathit{msgs} \cup \mathit{mgs}')) = \mathit{arb}
   \end{array}}
     {(\mathit{data},\mathit{msgs}) {\xrightarrow{(m,a,b,r,\mathit{arb})}} (\mathit{data}[r:=d], \mathit{msgs} \cup \mathit{mgs}')}
 \end{array}$

\medskip
    $\begin{array}{l c} \bigfrac{ \mathit{msg} = (\_,\_,r) \in \mathit{msgs},(\mathit{data}(r),r,\mathit{msg}) \rightarrow_m d } {(\mathit{data},\mathit{msgs}) {\xrightarrow{\mathit{apply}(\mathit{msg})}} (\mathit{data}[r:=d],\mathit{msgs} - \{ \mathit{msg} \})} \end{array}$    
  \end{center}
\medskip
Above, $\mathit{data}[r:=d]$ denotes the function $\mathit{data}$ where only the local state of $r$ is changed to $d$.
\item[-] $q_0 = (data_0,\emptyset)$ is the initial state, $data_0$ maps each replica identifier into $d_0$.
\end{itemize}

A trace of $[|\mathit{Imp}|]$ satisfies causal delivery if given $o_1 <_{\mathit{hb}} o_2$, the message generated by $o_2$ can be applied on a replica only if the message generated by $o_1$ has already been applied on this replica. The formalization of a version of $[|\mathit{Imp}|]$, denoted by $[|\mathit{Imp}|]_{cd}$, that satisfies causal delivery is given in Appendix \ref{sec:appendix definitions and proofs of section implementation}. Each state of $[|\mathit{Imp}|]_{cd}$ stores also a happens-before order between messages in $\mathit{msgs}$ sent to the same replica and the $\mathit{apply}(\mathit{msg})$ transitions are enabled only when $\mathit{msg}$ is minimal in this happens-before order.



\noindent {\bf Example 6 [OR-set algorithm~\cite{Shapiro:2011}]} The OR-set algorithm tags each value added to the set with a unique identifier. A replica stores in a set $S$ the tagged values that are generated when the client submits an $add$ operation or which are received from other replicas. A $\mathit{rem}(a)$ operation submitted at a replica $r$ removes all tagged versions of $a$ which are visible to $r$ when the operation is submitted. This algorithm assume causal delivery of messages. Formally, this implementation is defined by $(\rightarrow_o,\rightarrow_m,d_0,f_{\mathit{arb}}, \Sigma_e)$ where


\begin{itemize}
\setlength{\itemsep}{0.5pt}
\item[-] $\mathbb{L} = \{ (S,\mathit{ctr}) \vert S \subseteq \mathbb{D} \times (\mathbb{R} \times \mathbb{N}), \mathit{ctr} \in \mathbb{N} \}$. The elements of $S$ are of the form $(a,\mathit{id})$, and $\mathit{ctr}$ is a per-replica counter used to generate unique identifiers $\mathit{id}$ for values.

\item[-] $(S,\mathit{ctr}) \xrightarrow[\mathit{mgs} = \{ (a,\mathit{id},r,r') \vert r' \neq r \}]{(\mathit{add},a,r)}_o (S \cup \{ (a,\mathit{id}) \}, \mathit{ctr}+1)$ and $\mathit{id} = (r,\mathit{ctr})$.

\item[-] $(S,\mathit{ctr}) \xrightarrow[\mathit{mgs} = \{ (a,S'',r,r') \vert r' \neq r \} ]{(\mathit{rem},a,r)}_o (S \setminus \{ (a,\mathit{id}) \in S \}, \mathit{ctr})$, $S'' = \{ \mathit{id} \vert (a,\mathit{id}) \in S \} \neq \emptyset$.

\item[-] $(S,\mathit{ctr}) \xrightarrow[\mathit{mgs} = \emptyset]{(\mathit{lookup},a,\mathit{true},r)}_o (S, \mathit{ctr})$, if there exists $\mathit{id}$ such that $(a,\mathit{id}) \in S$.

\item[-] $(S,\mathit{ctr}) \xrightarrow[\mathit{mgs} = \emptyset]{(\mathit{lookup},a,\mathit{false},r)}_o (S, \mathit{ctr})$, if $(a,\mathit{id}) \notin S$ for each $\mathit{id}$.

\item[-] $(S,\mathit{ctr}) \xrightarrow{\mathit{apply}((a,\mathit{id},\_,r))}_m (S \cup \{ (a,\mathit{id}) \}, \mathit{ctr})$.

\item[-] $(S,\mathit{ctr}) \xrightarrow{\mathit{apply}((a,S',\_,r))}_m (S \setminus \{ (a,\mathit{id}) \vert \mathit{id} \in S' \}, \mathit{ctr})$.

\item[-] $d_0 = (\emptyset,0)$.

\item[-] $\Sigma_e$ is the union of (1) $(a,\mathit{id})$ in either $\mathit{data}[\_].S$, or $(a,\mathit{id},\_,\_) \in \mathit{msgs}$, or $\exists (a,S',\_,\_) \in \mathit{msgs}$ and $\mathit{id} \in S'$, and (2) $(a,S',r)$ in $\mathit{msgs}$.

\item[-] $f_{\mathit{arb}}$ always return $\emptyset$.
\end{itemize}




\noindent {\bf Example 7 [Replicated growable array (RGA, for short)~\cite{Attiya:2016}]} Each replica stores a set $N$ of tuples $(a,t,p)$, where $a$ is an input value added to the array, $t$ is its unique timestamp, and $p < t$ is the timestamp of its ``parent''. The elements of $N$ are required to form a tree by following the ``parent'' field (a special time stamp value $\circ$ represents the root).
%Let $\circ$ be a special time stamp value representing the root. The elements of $N$ are required to form a time stamped insertion tree (TI tree, for short), while the parent of each element in $N$ is still in $N$, each element of $N$ can tract to root by several times of checking parent field, and the time stamp of parent is less than its descendants.
%We can see that the union of multiple TI tree is also a TI tree.
Each replica also manages a set of ``tombstones'' $T$, which are identifiers of values that were removed from the array. {\color {red} $\mathit{trav}(N,T)$ is the list of this replica, and is obtained by traversing $N$ in prefix order (children are visited in decreasing time stamp order) and keeping only identifiers of nodes that are not in $T$. To insert $a$ into position $\mathit{pos}$, we put $(a,t,p)$ into $N$, where $p = \mathit{trav}(N,T)[\mathit{pos}]$, and $t$ is larger than anyone of $N$. This make $(a,t,p)$ the first child of $p$ in the tree of $N$. We remove $a$ by putting it into $T$. The process of applying $\mathit{add}$ and $\mathit{rem}$ message is similar. We define an order $<_{\mathit{ts}}$ between time stamps by $(x_1,r_1) <_{\mathit{ts}} (x_2,r_2)$ iff $x_1 < x_2 \vee (x_1 = x_2 \wedge r_1 < r_2)$. Let $\mathit{ts}(N)$ be the set of timestamps in $N$.
%and let $\mathit{trav}(N,T)$ be the sequences obtained by traversing $N$ in prefix order (children are visited in decreasing timestamp order) and keeping only identifiers of nodes that are not in $T$.
}








%Replicated growable array (RGA, for short) assumes causal delivery. A time stamp is a tuple $(x,r)$ where $x \in \mathbb{N}$ and $r \in \mathbb{R}$. The order of time stamp is as follows: $(x_1,r_1) <_{\mathit{ts}} (x_2,r_2)$, if $x_1 < x_2 \vee (x_1 = x_2 \wedge r_1 < r_2)$. A node $n$ is a tuple $(a,t,p)$, where $a \in \mathbb{D}$ is the value (identifier) of $n$, $t$ is the time stamp of $n$, and $p$ is the parent of $n$, either a time stamp used to identify or a special value $\circ$ that representing a pre-existed root node. We write $(b,p,p') \xrightarrow{\mathit{pa}} (a,t,p)$ to represent the former is the parent of the latter. A set $S$ of node is a time stamped insertion tree (TI tree, for short), if (1) the time stamp and identifier of each element is unique in $S$, (2) if $(a,t,p) \in S$, then $S$ contains an element with time stamp $p$, (3) for each element $x$ in $S$, we have $\circ \xrightarrow{\mathit{pa}}^* x$, and (4) time stamp of the parent is less than the parent of its descendants. We can see that the union of multiple TI tree is also a TI tree. In RGA algorithm, each replica manage a TI tree $N$ and a tombstone $T$, which are set of identifiers. Let $\mathit{ts}(N)$ returns the set of time stamp in $N$, let $\mathit{trav}(N,T)$ be the sequences obtained by traversing $N$ with prefix order (visit the children in decreasing time stamp order) and then returns identifiers of nodes that are not in $T$.

\begin{itemize}
\setlength{\itemsep}{0.5pt}
\item[-] $\mathbb{L} = \{ (N,T) \vert N$ is a TI tree, $T \in \mathbb{D} \}$.

\item[-] If $(a,\_,\_) \notin N$, then $(N,T) \xrightarrow[\mathit{mgs} = \{ (\mathit{add},a,t,p,r,r') \vert r' \neq r \}]{(\mathit{add},a,\mathit{pos},r)}_o (N \cup \{ (a,t=(\mathit{max}(\mathit{ts}(N))+1,r),p) \}, T)$. If $\mathit{pos} \neq 0$, then $p = \mathit{trav}(N,T)[\mathit{pos}]$; Else, $p = \circ$.

\item[-] If there exists $(a,t,p) \in N \wedge (a,\_,\_) \notin T$, then $(N,T) \xrightarrow[\mathit{msgs} = \{ (\mathit{rem},a,r,r') \vert r' \neq r \} ]{(\mathit{rem},a,r)}_o (N,T \cup \{ a \})$.

\item[-] $(N,T) \xrightarrow[\mathit{mgs} = \emptyset]{(\mathit{read},\mathit{trav}(N,T),r)}_o (N,T)$.

\item[-] $(N,T) \xrightarrow{\mathit{apply}((\mathit{add},a,t,p,\_,r))}_m (N \cup \{ (a,t,p) \},T)$.

\item[-] $(N,T) \xrightarrow{\mathit{apply}((\mathit{rem},a,\_,r))}_m (N,T \cup \{ a \})$.

\item[-] $d_0 = (\emptyset,0)$.

\item[-] $\Sigma_e$ is the union of (1) identifier $(\mathit{add},a,t)$ in $N$, $T$, and in $(\mathit{add},a,\_,\_,\_,\_) \in \mathit{msgs}$, and (2) $(\mathit{rem},a,r)$ for $(\mathit{rem},a,r,\_) \in \mathit{msgs}$.
\item[-] $f_{\mathit{arb}}$ return a sequence that is obtained as follows: Let $S$ be the union of $\mathit{data}[r].N$ for each replica. $f_{\mathit{arb}}$ traverses $S$ with prefix order visit the children in decreasing time stamp order, and returns identifiers of nodes.
\end{itemize}






























\forget{
\noindent {\bf Example 7. woot algorithms \cite{Oster:2006}}: Woot algorithm does not assume causal delivery. A W-character $x$ is a five-tuple $(\mathit{id},a,v,\mathit{id}_p,\mathit{id}_n)$, where $\mathit{id} \in \mathbb{R} \times \mathbb{N}$ is the identifier of $x$; $a$ is the value of $x$; $v \in \{ \mathit{true},\mathit{false} \}$ indicates whether $x$ is visible or tombstone; $\mathit{id}_p$ and $\mathit{id}_n$ are the identifier of previous and next W-character of $x$, while each W-character is generated to be inserted in between two W-characters.

\begin{itemize}
\setlength{\itemsep}{0.5pt}
\item[-] $\mathbb{L} = \{ (\mathit{ctr}, \mathit{str}) \vert \mathit{ctr} \in \mathbb{N}, \mathit{str}$ is a string of W-character $\}$. $\mathit{str}$ is the list seen by this replica, and $\mathit{ctr}$ is per-replica counter to make unique identifier. Let $<_{\mathit{ts}}$ be the order of time stamp, such that $(r_1,c_1) <_{\mathit{ts}} (r_2,c_2)$, if $(c_1 < c_2) \vee (c_1 = c_2 \wedge r_1 < r_2)$. Given a sequence $s$, let $<_s$ be order of $s$, such that $a <_s b$, if they are both in $s$ and $a$ has smaller index.

{\color {red}
\item[-] $(\mathit{ctr},\mathit{str}) \xrightarrow[\mathit{msgs} = \{ (y,r,r') \vert r' \neq r \}]{(\mathit{rem},x,r)}_o (\mathit{ctr},\mathit{str}')$, if there exists $y = (\_,x,\textit{true},\_,\_) \in \mathit{str}$, and $\mathit{str}'$ is obtained from $\mathit{str}$ by changing the flag of $y$ into $\mathit{false}$.

\item[-] $(\mathit{ctr},\mathit{str}) \xrightarrow[\mathit{msgs} = \emptyset]{(\mathit{read},l,r)}_o (\mathit{ctr},\mathit{str})$, where $l$ is the sequence of values of elements in $\mathit{str}$ with flag $\mathit{true}$.

\item[-] $(\mathit{ctr},\mathit{str}) \xrightarrow[\mathit{msgs} = \{ (w_3,w_1,w_2,r,r') \vert r' \neq r \}]{(\mathit{add},x,\mathit{pos},r)}_o (\mathit{ctr}+1,\mathit{str}')$: Assume $\mathit{str}[\mathit{pos}]=w_1$ and $\mathit{str}[\mathit{pos}+1]=w_2$, then let $w_3 = ((r,\mathit{ctr}),x,\mathit{true},w_1.\mathit{id},w_2.\mathit{id})$, and $\mathit{str}' = \mathit{insert}(w_3,w_1,w_2)$ for the recursive function defined in Algorithm \ref{alg:insert}. Notions of Algorithm \ref{alg:insert} are as follows: Given $d_u = (\mathit{id},a,v,\mathit{id}_p,\mathit{id}_n)$, $\mathit{pre}(d_u) = \mathit{id}_p$ and $\mathit{next}(d_u)=\mathit{id}_n$; $\mathit{subseq}(\mathit{str},c_p,c_n)$ returns the sub-sequence of $\mathit{str}$ between $c_p$ and $c_n$; $\mathit{pos}(\mathit{str},c_n)$ returns the position of $c_n$ in $\mathit{str}$; $\mathit{ins}(\mathit{str},c,\mathit{pos})$ inserts $c$ into $\mathit{str}$ at position $\mathit{pos}$.

\item[-] $(\mathit{ctr},\mathit{str}) \xrightarrow{\mathit{apply}((\_,a,\_,\_,\_))}_m (\mathit{ctr},\mathit{str}')$, if $\mathit{str}$ contains W-character $w$ with value $a$, and $\mathit{str}'$ is obtained from $\mathit{str}$ by changing the flag of $w$ into $\mathit{false}$.

\item[-] $(\mathit{ctr},\mathit{str}) \xrightarrow{\mathit{apply}((w_1,w_2,w_3,\_,\_))}_m (\mathit{ctr},\mathit{str}')$, if $\mathit{str}$ contains $w_2$ and $w_3$, and $\mathit{str}' = \mathit{insert}(w_1,w_2,w_3)$ for the recursive function defined in Algorithm \ref{alg:insert}.
}

\item[-] $d_0 = (0,\epsilon)$.

\item[-] $\Sigma_e$ is the union of (1) W-character $w$ in local states and messages, and (2) $(w_1,w_2,w_3,r,\_)$ in messages.

\item[-] {\color {red}$f_{arb}$ is obtained by applying $\mathit{insert}$ to all W-character in $\mathit{str}$ from $\epsilon$. \cite{Oster:2006} proves convergence. We give another convergence proof in Appendix \ref{sec:appendix definitions and proofs of section implementation}. We prove that, if operations in $S$ are convergent and $o_1$, $o_2$ replies on $S$, then applying $S \cup \{ o_1,o_2 \}$ also convergent.}

%$f_{arb}$ orders $x$ and $y$ as follows: Given $x = (id,a,v,id_p,id_n)$, we say that $x$ relies on W-character with identifiers $id_p$ and $id_n$. We say that $(x,y) \in f_{arb}$, if one of the following cases holds: (1) there exists $z_1,\ldots,z_k$, such that $x=z_1$, $y=z_k$, and for each $u$, $next(z_u)=z_{u+1} \vee pre(z_{u+1}) = z_u$. (2) Starting from $\epsilon$, we put all elements that $y$ and $z$ rely, then we put $y$ and $z$, and find that $y$ is before $z$ in $str$ of the resulting local state.


\end{itemize}

\begin{algorithm}[t]
\KwIn {$c$, $c_p$, $c_n$}
\KwOut{a W-character string}

Let $s' = \mathit{subseq}(\mathit{str},c_p,c_n)$ \;

\If {$s' = \epsilon$}
{
    \Return $\mathit{ins}(\mathit{str},c,\mathit{pos}(\mathit{str},c_n))$\;
}
\Else
{
    let $L = c_p \cdot d_0 \cdot \ldots \cdot d_k \cdot c_n$, where $d_0,\ldots,d_k$ are the W-char $d_u$ in $s'$ such that $\mathit{pre}(d_u) \leq_{\mathit{str}} c_p$ and $c_n \leq_{\mathit{str}} \mathit{next}(d_u)$\;
    let $j = 1$\;
    {\bf while} $(j < \vert L \vert -1 \wedge L[j] <_{\mathit{ts}} c)$\ {\bf do}
    {
        $j = j+1$\;
    }
    return $\mathit{insert}(c,L[i-1],L[i])$\;
}
\caption{$\mathit{insert}$}
\label{alg:insert}
\end{algorithm}
}

%%% Local Variables:
%%% mode: latex
%%% TeX-master: "draft"
%%% End:
