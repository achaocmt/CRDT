%!TEX root = draft.tex

\section{Definitions and Proofs of Section \ref{sec:reference implementation}}
\label{sec:appendix definitions and proofs of section reference implementation}



\subsection{Proof of Lemma \ref{lemma:executions of reference implementation are eventual consistent}}
\label{subsec:appendix proof of lemma executions of reference implementation are eventual consistent}

{\noindent \bf Lemma \ref{lemma:executions of reference implementation are eventual consistent}}: $\forall t \in \llbracket RImp(Spec) \rrbracket$, $poSet(t)$ is eventual consistent w.r.t $Spec$.

\begin {proof}

Assume $RImp(Spec) = (Q,\Sigma,vis,,q_0,li,\rightarrow,livReq)$, let $t = \alpha_1, \ldots, $ and $\exists q_1,\ldots$, such that $q_0 {\xrightarrow{\alpha_1}} q_1 {\xrightarrow{\alpha_2}} \ldots$, and $livReq(t) = \textit{true}$. Let $poSet(t)=(O_t,<_t)$. Let $gi = (O_t,<_{gi})$ be the global interpretation of $t$.

For each operation $o$, assume $o$ is first introduced by a transition from $q_i$ to $q_{i+1}$, then, the local interpretation of $o$ is $li(q_i,r)$, where $r$ is the replica of $o$.

We prove this lemma by consider the four requirements individually:

\begin{itemize}
\setlength{\itemsep}{0.5pt}
\item[-] $\textit{GIpf}$: Let us consider two kinds of operations $o$.

If $\exists o'$, such that $o' <_{gi} o$, then $\exists i < j$, such that $o'$ and $o$ are the operations of $\alpha_i$ and $\alpha_j$, respectively, and one of the following cases holds: (1) $o'$ and $o$ are of same replica, or (2) $\exists i', i < i' < j$, $\alpha_{i'}=addDel(o',r')$ and $r'$ is the replica of $o$. In each case, it is easy to see that $<^{-1}_{gi}(o)$ contains lee or equal operations than the set of operations in $t[1,j]$ and its number is obviously finite.

    If $\neg \exists o'$, such that $o' <_{gi} o$, then it is easy to see that $o$ is the first operation of its replica and no operation is delivered to that replica before $o$. In this case, it is obvious that $<^{-1}_{gi}(o)$ is finite.

\item[-] $\textit{THINAIR}$: According to the definition, we could see that on each state $q_i$, local interpretation is a subset of visibility relation, and $ro$ is also a subset of visibility relation. It is easy to prove that on each state, $vis$ is acyclic.

\item[-] $\textit{RVAL}$: We only need to consider query operations for $\textit{RVAL}$ property, since only query operations have return values. According to construction of $\llbracket RImp(Spec) \rrbracket$, to launch a query operation $(m,a,b,rid,oid)$ transition of replica $r$ from state $q_i$, we already check whether $li(q_i,r)$, the local interpretation w.r.t $q_i$ and $r$, is in $Spec(m,a,b)$.

\item[-] $\textit{EVENTUAL}$: Let $P=(O_P,<_P)$ be a finite prefix of $gi$.

We can see that $<_P$ is the visibility relation of operations in $O_P$. Let $O'_P = \{ o \vert \exists o_1,o_2 \in O_P,$ $o_1$ is visible to $o_2$ via $o'_1,\ldots,o'_m$, and $o \in \{ o'_1,\ldots,o'_n \} \}$. It is easy to see that $O'_P$ is finite, and assume that operations of $O_P \cup O'_P$ are chosen before $\alpha_{idx}$

Since $livReq(t)=true$, $\exists idx1$, such that in $t[1,idx1]$, all operations in $t[1,idx]$ has been delivered to every replica. According to our construction of $li$, this implies that for each state after $\alpha_{idx1}$ and each replica (1) its local interpretation contains $O_P$ and $O'_P$, and (2) the relation of local interpretation then contains the visibility relation between operations in $O_P$.
\end{itemize}

This completes the proof of this lemma. $\qed$
\end {proof}








\section{Definitions and Proofs of Section \ref{sec:implementation}}
\label{sec:appendix definitions and proofs of section implementation}



{\noindent \bf Lemma \ref{lemma:semantics of imp cd contains the set of causal delivery executions of semantics of imp}}: $\llbracket imp \rrbracket_{cd} = \{ t \vert t \in \llbracket imp \rrbracket \wedge t$ satisfies causal delivery $\}$.

\begin {proof}

Given $t = \alpha_1 \cdot \ldots \in \llbracket imp \rrbracket_{cd}$, if there exists $q_1,\ldots \in Q$, such that $q_0 {\xrightarrow{\alpha_1}} q_1 {\xrightarrow{\alpha_2}} \ldots$. We need to prove the following property: On each state $q_i=(repD_i,msgs_i,<_i)$,

\begin{itemize}
\setlength{\itemsep}{0.5pt}
\item[-] $P_1$: $msgs_i = \{ (dat,r,r') \vert$ the operation generating this message is launched by replica $r$ and its message has not been applied to replica $r'$ in $t[1,i]\}$.

\item[-] $P_2$: $m_1 <_i m_2$, iff $m_1$ and $m_2$ have same destination replica, and the operation generating $m_1$ happens before the operation generating $m_2$ in $t[1,i]$.

\item[-] $P_3$: $m <_i r$, iff the destination replica of $m$ is not $r$, and the operation generating $m$ is visible to replica $r$ in $t[1,i]$, and does not happen before ``any operation that launched by replica $r$'' in $t[1,i]$.
\end{itemize}

Once we prove that $P_1$, $P_2$ and $P_3$ holds for each $q_i$, let us prove this lemma by contradiction: Assume that update operations $o_1 <_{hb} o_2$, messages of $o_1$ are $\{ (d_1,r_1,\_) \}$, messages of $o_2$ are \{ $(d_2,r_2,\_) \}$, $\alpha_i = apply((d_2,r_2,r'),r')$, $\alpha_j = apply((d_1,r_1,r'),r')$, and $i<j$. Then in $q_i=(repD_i,msgs_i,<_i)$, since $o_1 <_{hb} o_2$ and transition rules of $\llbracket imp \rrbracket_{cd}$, we could not launch $apply((d_2,r_2,r'),r')$ transition, which is the contradiction.



Let us begin to prove that each state $q_i=(repD_i,msgs_i,<_i)$ satisfies properties $P_1$, $P_2$ and $P_3$. We prove this by induction on $t$.

\begin{itemize}
\setlength{\itemsep}{0.5pt}
\item[-] It is obvious that $q_0$ satisfies properties $P_1$, $P_2$ and $P_3$.

\item[-] Since $\alpha_1$ must be either a query transition or an update transition, it is easy to see that $<_1 = \emptyset$ and $q_1$ satisfies properties $P_1$, $P_2$ and $P_3$.

\item[-] Assume that $q_i=(repD_i,msgs_i,<_i)$ satisfies properties $P_1$, $P_2$ and $P_3$. let us consider $q_{i+1}= (repD_{i+1},msgs_{i+1},<_{i+1})$,

    \begin{itemize}
    \setlength{\itemsep}{0.5pt}
    \item[-] If $q_{i+1}$ is obtained from $q_i$ by a query transition, then this holds obviously.

    \item[-] Else, if $q_{i+1}$ is obtained from $q_i$ by a update transition. Then we can see that $(repD_i,msgs_i,<_i) {\xrightarrow{m(a,b,r)}} (repD_{i+1},msgs_{i+1}=msgs \cup \{ (dat,r,r') \vert  r' \in RId \wedge r \neq r' \},<_{i+1} = <_i \otimes dat)$.

        Since we add messages $\{ (dat,r,r') \vert  r' \in RId \wedge r \neq r' \}$ into $msgs_i$, we can see that $P_1$ holds.

        To satisfy $P_2$ and $P_3$, we need to make newly add messages maximal w.r.t $<$ and still keep transitivity, which is done by $<_i \otimes dat$.

    \item[-] Else, $q_{i+1}$ is obtained from $q_i$ by a applying transition. Then we can see that $(repD_i,msgs_i,<_i) {\xrightarrow{apply(m=(dat,r_1,r))}} (repD_{i+1},msgs_{i+1} = msgs_i - \{ m \}, <_{i+1} = <_i \otimes m )$, where $m$ is minimal w.r.t $<_i$ among messages in $msgs_i(r)$.

        Since we use one message $m$ in this process and remove it from $msgs_i$, we can see that $P_1$ holds.

        Applying message will introduce new visibility relation. To satisfy $P_2$ and $P_3$, we need to first forget $m$, record the newly introduced visibility relation, and still keep transitivity. This is done by $<_i \otimes m$.
    \end{itemize}
\end{itemize}

This completes the proof of this lemma. $\qed$
\end {proof}





\section{Definitions and Proofs of Section \ref{sec:simulation relation}}
\label{sec:appendix definitions and proofs of section simulation relation}


{\noindent \bf Theorem \ref{theorem:equivalence of our simulation relation and sequence inclusion}}: $RImp(Spec)$ trace refines $\llbracket imp \rrbracket$, if and only if there exists a simulation relation between $\llbracket imp \rrbracket$ and $RImp(Spec)$. 

\begin {proof} 
The $\textit{only if}$ direction is obvious. 

Let us prove the $\textit{if}$ direction. By assumption we already have that $RImp(Spec)$ trace refines $\llbracket imp \rrbracket$. Assume $\llbracket imp \rrbracket = (Q_{imp},\Sigma_{imp},\rightarrow_{imp},q_{0imp})$ and $RImp(Spec) = (Q_s,\Sigma_s,vis,q_{0s},li,\rightarrow_s,livReq)$. A relation $R \subseteq Q_{imp} \times Q_s$ is defined as follows: $(q_i,q_s) \in R$, if $\exists t_{imp} = \alpha_1 \cdot \ldots \cdot \alpha_k,t_s = \beta_1 \cdot \ldots \cdot \beta_k, q_{1imp}, \ldots, q_{kimp}, q_{1s}, \ldots, q_{ks}$, such that $q_{0imp} {\xrightarrow{\alpha_1}}_{imp} q_{1imp} \ldots {\xrightarrow{\alpha_k}}_{imp} q_{kimp}$ is an execution of $\llbracket imp \rrbracket$, $q_{kimp}=q_i$, $q_{0s} {\xrightarrow{\beta_1}}_s q_{1s} \ldots {\xrightarrow{\beta_k}}_s q_{ks}$ is an execution of $RImp(Spec)$, $q_{ks}=q_s$, and $t_{imp}$ and $t_s$ correspond. Let us prove that $R$ is a simulation relation. Note that, given $t_{imp}$, there exists at most one $t_s$, such that $t_{imp}$ and $t_s$ correspond. 

\begin{itemize}
\setlength{\itemsep}{0.5pt}
\item[-] If $q_{imp} {\xrightarrow{m(a,b,r)}}_{imp} q'_{imp}$: It is easy to see that $t'_{imp} = t_{imp} \cdot m(a,b,r) \in \llbracket imp \rrbracket$. 

Since $RImp(Spec)$ trace refines $\llbracket imp \rrbracket$, there exists $t'_s$ of $RImp(Spec)$, such that $t'_{imp}$ and $t'_s$ correspond. 

Since given a trace $t$ of $\llbracket imp \rrbracket$, there exists at most one trace $t'$ of $RImp(Spec)$, such that $t$ and $t'$ correspond. It is easy to see that such $t'_s$ is unique. 

Since $t_{imp}$ and $t_s$ correspond and $RImp(Spec)$ is deterministic, we can see that $t'_s = t' \cdot m(a,b,r)$ is an execution of $RImp(Spec)$. Let $q_{ks} {\xrightarrow{m(a,b,r)}}_s q'_s$ It is easy to see that $t'_{imp}$ and $t'_s$ correspond. Therefore, $(q'_{imp},q'_s) \in R$. 

\item[-] If $q_{imp} {\xrightarrow{apply(m)}}_{imp} q'_{imp}$: It is easy to see that $t'_{imp} = t_{imp} \cdot apply(m) \in \llbracket imp \rrbracket$.

Since $RImp(Spec)$ trace refines $\llbracket imp \rrbracket$, there exists $t'_s$ of $RImp(Spec)$, such that $t'_{imp}$ and $t'_s$ correspond.

Since given a trace $t$ of $\llbracket imp \rrbracket$, there exists at most one trace $t'$ of $RImp(Spec)$, such that $t$ and $t'$ correspond. It is easy to see that such $t'_s$ is unique. 

We already know that $t_{imp}$ and $t_s$ correspond and $RImp(Spec)$ is deterministic. It is not hard to prove that, in $q_{kimp}$ and $q_{ks}$, $\forall r_1,r_2 \in RId$, the number of messages of $q_{kimp}$ which ``use source replica $r_1$ and destination replica $r_2$ and are still not applied'' is same as the number of operation of $q_{ks}$ which ``happens on replica $r_1$ and not visible to replica $r_2$''. Therefore, there exists transition $q_{ks} {\xrightarrow{addDel(o,r)}}_s q'_s$. Here $o$ and $r$ are obtained as follows: Assume after doing $t_{imp}$, $m=(\_,r',r)$ is the $i-th$ among ``messages with source replica $r'$, destination replica $r$ and still not applied'' w.r.t the occurring order of $t_{imp}$. Then, after doing $t_s$, $o$ is the $i-th$ among ``operations which happens on replica $r'$ and not visible to replica $r$'' w.r.t the occurring order of $t_s$. Let $t'_s = t' \cdot addDel(o,r)$. 

Let us prove that such $m$, $o$ and $r$ satisfies the requirements in definition of simulation relation. Since we already know that $q_{0imp} {\xrightarrow{t_{imp}}}_{imp}^* q_{kimp}$ and $q_{0s} {\xrightarrow{t_s}}_s^* q_{ks}$, it is not hard to prove that, the $i-th$ message among ``messages with source replica $r'$, destination replica $r$ and still not applied'' w.r.t the occurring order of $t_{imp}$ is the same as the $i-th$ among messages of $q_{kimp}$ with source replica $r'$ and destination replica $r$ w.r.t $<_{sd}$. Similarly, we can prove that, the $i-th$ operations among ``operations which happens on replica $r'$ and not visible to replica $r$'' w.r.t the occurring order of $t_s$ is the same as the $i-th$ among operations of $q_{ks}$ which happens on replica $r'$ and not visible to replica $r$ w.r.t $ro$. Therefore, $o$ and $r$ satisfies the requirements in definition of simulation relation. 

Let us prove that $(q'_{imp},q'_s) \in R$. From above discussion, it is not hard to see that $t'_{imp}$ and $t'_s$ correspond. Therefore, $(q'_{imp},q'_s) \in R$. 
\end{itemize} 

This completes the proof of this lemma. $\qed$
\end {proof}
