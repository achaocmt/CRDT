%!TEX root = draft.tex

\section{Preliminary}
\label{sec:preliminary} 

In this section, we introduce the definition of eventual consistency, and related notation. 

 

In this paper, we intend to use the definition of eventual consistency in \cite{Bouajjani:2014}. Simulation is a way to prove eventual consistency. However, it is hard to use simulation to prove eventual consistency in \cite{Bouajjani:2014}. The reason is that, the definition of eventual consistency in \cite{Bouajjani:2014} covers both a correctness part and a liveness part, while simulation concerns only one step of transitions and seems not able to state properties of infinite length (liveness part).

Intuitively, out method is as follows:

\begin{itemize}
\setlength{\itemsep}{0.5pt}
\item[-] To prove with simulation, we use a operational-style specification $\textit{OPSpec}$ of object, which covers a subset of possible and correct executions of a object.

Here, $\textit{OPSpec}$ explicitly record deliver of messages. If necessary, such as in the case of list specification, other necessary information can be also recorded.

\item[-] In \cite{Bouajjani:2014}, to deal with speculation, each operation must have its local interpretation. To ensure convergence, there is a global interpretation and it is required that, for each operation, only finite number of local interpretation diverge from global operation.

    In our method, since we only concern CRDT, where speculation is not used. We require that, if operation $o$ and $o'$ are of same replica and $o$ is earlier, then the local interpretation of $o$ is contained in the local interpretation of $o'$.

\item[-] The local interpretation of replica $k$, $li(k)$, depends only on $rb$ relation and $del$ relation.

\item[-] If $lo(k)$ and $lo(k')$ contains the same set of operations, then they answer same question for each read-only operation. Or we can say, they are in the same abstract state.
\end{itemize}

In short, we use the third, forth and fifth condition to get rid of global interpretation in \cite{Bouajjani:2014}.



