\section{Related Work}
\label{sec:related-work}

To the best of our knowledge, we are the first to provide a generic
operational framework for the characterization of CRDT specifications
based on the axiomatic characterizations of~\cite{Burckhardt:2014}
under diverse delivery assumptions of the underlying network.
%

There have been many efforts in formalizing the semantics of
replicated data
types~\cite{Burckhardt:2014,Burckhardt:2014b,ZellerBP14,MukundRS15}.
%
Unlike the approach in~\cite{Burckhardt:2014,Burckhardt:2014b}, our
proofs are based on standard simulation arguments, common to proofs in
programming languages.
%
Moreover, unlike~\cite{Burckhardt:2014,Burckhardt:2014b}, we provide
operational variants of the specifications which are instrumental in
proving the correctness of implementations.

In~\cite{ZellerBP14}, the proofs provided are based on state-based
implementations of CRDTs, our work is concerned with operation-based
implementations.

An important difference between our work and the aforementioned is
that to be able to express a simulation relation, we need to export at
the level of the operational specification the delivery assumptions of
the system.
%
In turn this means that our proofs are with respect to a more concrete
specification than these other works, since the delivery assumptions
are not apparent in their specifications.
%
We will consider in future work how to prove that out reference
semantics preserves the specifications \'a la~\cite{Burckhardt:2014}
where the delivery conditions have been removed.
%

Perhaps the work that is closer to ours is~\cite{MukundRS15,MukundRS15b} 
since they also use FSM-based specifications. 
Unlike those works, our specifications of CRDTs are given in the
format of~\cite{Burckhardt:2014} and the reference implementing is
obtained thereof.
%
Moreover, while they propose a CEGAR methodology to carry out the
proofs of the data type implementations, our proofs are based on
simulation arguments.
%
Finally, they do not provide proofs of concrete algorithms in their
work. 

There are works on the verification of program properties about
Credits~\cite{GotsmanYFNS16}.
%
We consider our work to be orthogonal~\cite{GotsmanYFNS16} since we
are proving the correctness of implementations w.r.t. the
specification. 

%%% Local Variables:
%%% mode: latex
%%% TeX-master: "draft"
%%% End:
