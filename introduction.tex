%!TEX root = draft.tex
\section{Introduction}
\label{sec:introduction}

Conflict-Free Replicated Data Types (CRDTs)~\cite{ShapiroPBZ11} have
emerged as the most promising solution to the problem of ensuring
some degree of consistency in highly-available distributed systems.
%
The key idea is that while a CRDT might not be consistent at all times
thought the whole system during an execution, but the state of the
data type will \emph{eventually} converge to a unique state (c.f.
eventual consistency~\cite{Burkhardt}).
%
All the while, all participants of the systems can issue operations
which are immediately performed in a single replica without need to
synchronize with other participants, thus achieving high-availability,
even under partitions.

While CRDTs offer great advantages in terms of performance and
availability, their semantics and implementation are far from simple,
often involving sophisticated specifications~\cite{ShapiroPBZ11,
  Burkhardt}, and even more involved implementations.
%
As a testament to this claim, we remark that the discovery of a single
CRDT implementation often deserves a paper itself~\cite{RAG, WOOT,
  ...}.
%
A consequence of the complexity of CRDT implementations is that it is
often hard to gain confidence on their correctness.
%
In this paper we consider a generic methodology to the verification of
CRDTs implementing collection data structures.

\fxnote*{Not sure} { 
  Some of the most important data types in
  distributed programming are lists, tables, etc.
}
We consider the problem of formally verifying that a certain
implementation of a CRDT satisfies a formal specification given in the
style of~\cite{Burkhardt}.
%
Unlike prior approaches to the verification of CRDTs we
\fxwarning*{Complete}{what is it that is remarkably new here?}.
%
To that end, we develop the following methodology:
\begin{itemize}[$\bullet$]
\item Starting from a specification in the style of~\cite{Burkhardt},
  we provide a reference operational semantics of the CDT which is
  then used as the definitive correctness condition for any
  implementation of the data structure, 
\item we then show how we can simplify the specifications by removing
  unnecessary information from the specification whenever possible,
  this step can greatly reduce the complexity of the specifications,
  and thus be instrumental in the proofs to be carried out later. 
\item we take well known implementations of sophisticated CRDTs: WOOT,
  RGA, and show how simulation relations w.r.t. the reference
  semantics above-mentioned can easily be built for each of them, 
\item we show that the simulation above actually implies the
  correctness of the implementations. 
  % 
  This conclusively shows the conformance of the implementations with
  the specification. 
\end{itemize}

\fxfatal*{Do we?}{
While this paper deals only with a few CRDT implementations, we
believe that the approach is general enough to cover most CRDT
existing implementations.
%
To the best of our knowledge ...
}


\fxnote{GP: Should we compare to others here?}

\paragraph{{\bf Paper Outline}:} \fxnote{Todo}




%%% Local Variables:
%%% mode: latex
%%% TeX-master: "draft"
%%% End:
